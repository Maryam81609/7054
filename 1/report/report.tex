\documentclass[12pt]{article}

\newcommand{\code}[1]{\textsf{#1}}

\title{Design Decisions and Evaluation of Scanner}
\author{Maryam Dabaghchian-u1078006}
\date{\today}

\begin{document}
\maketitle

\section{Design Decisions}
For the scanner, I made the following design decisions:

\begin{itemize}
	\item[1.] Options:
		\begin{itemize}
			\item I use the option \code{IGNORE\_CASE = false} value, which is 
			the default value, to distinguish between lower case letters and 
			capital ones. For instance, the word \code{while} will be recognized 
			as a keyword or token 27, whereas the word \code{While} will be 
			recognized as an identifier or token 36 according to my 
			implementation.
			%
			\item By default \code{SANITY\_CHECK = true}, which makes JavaCC 
			perform different checks on the specification file, e.g., left 
			recursion, ambiguity and etc. check. Detecting such errors can 
			prevent the unexpected behavior in the generated scanner or parser.
			%
			\item I use the option \code{STATIC = false}, which allows instantiation of multiple scanners. This option is not useful or required in this level. The default value for this option is \code{true}.
			%
			\item That is was good to set the option \code{CACHE\_TOKENS = 
			true}, as our target programs are not interactive. With this option 
			the generated scanner or parser looks ahead for eaxtra tokens, that 
			improves performance. However, looking a head for extra tokens does 
			not work for interactive programs, as expected.
		\end{itemize}
		%	
	\item[2.] Rule Priority:
		\begin{itemize}
			\item I have defined keyword tokens prior to identifier token, so 
			that keywords such as \code{if}, \code{while}, and etc. will not 
			be recognized as identifiers, but keyword tokens.
%			%
%			\item I guess the reason for ordering the tokens as represented in \code{token\_codes.txt} file, is based on how often a token is being used in a program. For intance, the tokens \code{\{ \}} could be used  
		\end{itemize}
		%
	\item[3.] Since a line comment starts with \code{//} and after that any 
	character can appear to the end of line, I specified any character with 
	\code{($\sim["\backslash $n$","\backslash $r$"])$*}, which states a 
	combination of anything except end of line character ending with an end of 
	line character.
	%
	\item[4.] The \code{ant} builder automatically first generates the scanner, then compiles it, and finally scans the input file. It was better to separate the generating and compilation of the scanner as one target and scanning an input file as another target. 
\end{itemize}


\end{document}